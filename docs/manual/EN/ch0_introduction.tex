\chapter*{Introduction}
The idea for PySoar emerged during the national junior gliding competition at Venlo 2014. During this competition, the contestants got an extensive analysis of a specific competition day, enabling direct comparison of their flight with their competitors. Parameters such as average thermal strength, cruise speed and start height exposed the strengths and weaknesses of each flight. There were a view shortcomings to this analysis however. First of all the creation of the excel sheets was very time consuming since all the numbers had to be manually copied from SeeYou and flights could not be analyzed simultaneously. Furthermore, the interpretation of flight parameters was not always clear, causing misinterpretation and lack of insight.\\
\ \\
PySoar tries to solve these issues and enable fast and easy comparison of glider flights which are flown in a competition. It is based on the Python programming language and is completely stand alone, thereby ensuring cross platform compatibility and independence from expensive analysis tools. It starts with IGC files as published on the soaringspot website (automatically including the competition details) and delivers an excel sheet in which performance indicators can be compared immediately. Since the program has been built completely from scratch, the performance indicators are well understood and their meaning and calculation procedure will be published.